\documentclass[oneside]{book}
\usepackage[T1]{fontenc}
\usepackage[top=2.5cm,left=3.5cm,right=2.5cm,bottom=2cm]{geometry}
\usepackage[norsk]{babel}
\usepackage{titlesec}
\usepackage{times}
\usepackage[
    type={CC},
    modifier={by-sa},
    version={4.0},
]{doclicense}

\titleformat{\chapter}[display]
{\normalfont\huge\bfseries}{\chaptertitlename\ \thechapter}{20pt}{\Huge}

% this alters "before" spacing (the second length argument) to 0
\titlespacing*{\chapter}{0pt}{0pt}{40pt}

\title{Det industrielle samfunn og dets fremtid}
\author{Theodore Kaczynski\\Oversatt av Simon Nikolai Varøy}
\date{19.september 1995}

\begin{document}
\maketitle
\doclicenseThis\par
\tableofcontents

\chapter{Introduksjon}
\section*{1.}
Den industrielle revolusjon og dens konsekvenser har vært en katastrofe for
menneskeheten. Den har i høy grad økt forventet levealder for de av oss som bor
i ``utviklede'' land, men den har destabilisert samfunnet, gjort livet
utilfredsstillende, utsatt mennesker for uverdigheter, ført til omfattende
psykologisk lidelse (i den tredje verden også fysisk lidelse) og påført naturen
stor skade. Den kontinuerlige utviklingen av teknologi vil forverre
situasjonen. Den vil sannelig utsette mennesker for større uverdigheter og
påføre naturen større skade. Den vil også sannsynligvis føre til mer sosial
uro, psykologisk lidelse og muligens mer fysisk lidelse selv i ``utviklede''
land.

\section*{2.}
Det industrielt-teknologiske systemet kan overleve, men det kan også kollapse.
Om det overlever, er det en mulighet for at det vil oppnå en lav grad av fysisk
og psykologisk lidelse. Samtidig vil dette bare kunne skje etter en lang og
smertefull tidsperiode, og kostnaden vil være at mennesker og mange andre
levende organismer reduseres permanent til fabrikkerte produkter og brikker i
et større spill. Om systemet overlever, vil konsekvensene i tillegg være
uunngåelige: Det er ingen måte å reformere eller modifisere systemet slik at
man forhindrer det fra å frata folk verdighet og autonomi.

\section*{3.}
Om systemet kollapser vil konsekvensene fortsatt være svært smertefulle. Men jo
større systemet blir, jo verre vil konsekvensene av kollapsen bli. Så om
systemet skal kollapse, er det bedre at det skjer tidlig enn sent.

\section*{4.}
Vi tar derfor til orde for en revolusjon mot det industrielle systemet. Denne
revolusjonen kan ta i bruk voldsmetoder eller ikke og den kan være spontan
eller gå gradvis over noen tiår. Vi kan ikke forutse noe av dette, men vi
legger frem på en veldig generell måte hvilke metoder de som hater det
industrielle systemet bør ta i bruk for å legge til rette for en revolusjon mot
denne typen samfunnsstruktur. Dette vil ikke være en \emph{politisk} revolusjon.
Dens mål vil ikke være å styrte myndigheter, men heller det økonomiske og
politiske grunnlaget for dagens samfunn.

\section*{5.}
I denne artikkelen setter vi bare lys på noen av de negative utviklingene som
har kommet fra det industrielt-teknologiske systemet. Andre slike utviklinger
nevner vi bare kort eller ignorerer fullstendig. Dette betyr ikke at vi ser på
disse andre utviklingene som uviktige. Av praktiske hensyn må vi avgrense
diskusjonen vår til områder som har fått for lite offentlig oppmerksomhet eller
der vi har noe nytt å si.

\chapter{Psykologien bak moderne venstrevridd ideologi}
\section*{6.}
Nesten alle vil si seg enig i at vi lever i et svært problemfullt samfunn. En
av de mest omfattende manifestasjonene av denne galskapen i vår verden er
venstrevridd ideologi, så en diskusjon om psykologien bak denne ideologien kan
fungere som en introduksjon til diskusjonen om problemene med det moderne
samfunn generelt.

\section*{7.}
Men hva er venstrevridd ideologi? Under den første halvdelen av 1900-tallet
kunne man praktisk talt identifisert venstrevridd ideologi som sosialisme. I
dag er bevegelsen fragmentert og det er ikke klart hvem som kan defineres som
venstrevridd. Når vi snakker om venstrevridde i denne artikkelen tenker vi for
det meste på sosialister, kollektivister, ``politisk korrekte'' typer,
feminister, homo- og handikap aktivister, dyrerettsaktivister og lignende. Men
ikke alle som er assosiert med en av disse bevegelsene er venstrevridd. Det vi
sikter til når vi diskuterer venstrevridd ideologi er ikke i noen stor grad en
bevegelse eller ideologi, men heller en psykologisk type eller en samling av
relaterte typer. Derfor vil det vi mener med ``venstrevridd ideologi'' komme
frem tydeligere i løpet av vår diskusjon om venstrevridd psykologi. (Se avsnitt
227--30.)

\section*{8.}
Likevel vil vår forståelse av venstrevridd ideologi forbli en god del mindre
klar enn det vi skulle ønske, men det virker ikke som at det er noen måte å
fikse dette. Alt vi prøver å gjøre her er å indikere på en grov og omtrentlig
måte de to psykologiske tendensene som vi mener er den største drivkraften bak
moderne venstrevridd ideologi. Vi hevder på ingen måte at vi forteller
\emph{hele} sannheten om venstrevridd psykologi. Vår diskusjon er også bare ment
å gjelde for moderne venstrevridd ideologi. Spørsmålet om i hvor stor grad
diskusjonen vår gjelder for venstrevridde på 1800- og tidlig 1900-tall forblir
åpent.

\section*{9.}
De to psykologiske tendensene som underligger moderne venstrevridd ideologi
kaller vi ``følelser av underlegenhet'' og ``oversosialisering''. Følelser av
underlegenhet er karakteristisk for moderne venstrevridd ideologi helhetlig,
mens oversosialisering er karakteristisk for bare en viss del av ideologien;
men denne delen er svært innflytelsesrik.

\chapter{Følelser av underlegenhet}

\section*{10.}
Med ``følelser av underlegenhet'' tenker vi ikke bare på disse følelsene i en
streng forstand, men heller et helt spektrum av relaterte trekk; lav
selvtillit, maktesløshet, depressive tendenser, defaitisme, skyldfølelse,
selvhat osv. Vi argumenterer for at moderne venstrevridde ofte har slike
følelser (muligens mer eller mer undertrykt) og at disse følelsene vil bestemme
retningen til moderne venstrevridd ideologi.

\section*{11.}
Når noen tolker nesten alt som blir sagt om vedkommende (eller grupper som
vedkommende identifiserer seg med) som støtende, konkluderer vi med at han
eller hun føler på underlegenhet eller lav selvtillit. Denne tendensen er
utpreget blant minoritetsaktivister, uavhengig av om de tilhører gruppene de
forsvarer rettighetene til. De er hypersensitive når det kommer til ordene som
brukes for a beskrive minoriteter og alt som blir sagt om disse minoritetene.
Ordene ``neger'', ``orientalsk'', ``handikappet'' eller ``høne'' for en
afrikaner, asiater, funksjonshemmet person eller kvinne hadde historisk ingen
støtende konnotasjon. ``Berte'' eller ``høne'' var bare de feminine versjonene
av ``fyr'', ``type'' eller ``gubbe''. De negative konnotasjonene har blitt
tillagt disse ordene av aktivistene selv. Noen dyrerettsaktivister har gått så
lang som å avvise bruken av ordet ``kjæledyr'' og insisterer på at det skal
erstattes med ``dyrefølgesvenn''. Venstrelente antropologer strekker seg langt
for å ikke si noe om primitive kulturer som kan tolkes som negativt. De ønsker
å erstatte ordet ``primitiv'' med ``analfabet''. De virker nesten paranoide når
det kommer til ting som kanskje kan hentyde at en hvilken som helst primitiv
kultur er underlegen sammenlignet med vår. (Vi mener ikke å insinuere at
primitive kulturer \emph{er} underlegne. Vi påpeker bare hypersensitiviteten til
venstrelente antropologer.)

\section*{12.}
De som er mest sensitive når det gjelder ``politisk ukorrekt'' ordbruk er ikke
den gjennomsnittlige svarte ghettoinnbyggeren, asiatiske innvandreren,
kvinnelige overgrepsofferet eller funksjonshemmede, men heller et mindretall av
aktivister som i mange tilfeller ikke engang tilhører en ``undertrykt'' gruppe.
Disse har en privilegert bakgrunn. Politisk korrekthet står sterkest blant
universitetsprofessorer som har sikre jobber med komfortable lønninger, og
flertallet av dem er heterofile hvite menn fra middelklasse til øvre
middelklasse familier.

\section*{13.}
Mange venstrevridde identifiserer seg sterkt med problemene til grupper som
fremstår svake (kvinner), bekjempet (amerikanske indianere), frastøtende
(homofile) eller generelt underlegne. Venstrevridde føler selv at disse
gruppene er underlegne. De vil aldri innrømme til seg selv at de har disse
følelsene, men det er nettopp fordi de ser på disse gruppene som underlegne at
de identifiserer seg med problemene deres. (Vi mener ikke å si at kvinner,
indianere osv.\ \emph{er} underlegne; vi poengterer bare noe om venstrevridd
psykologi.)

\section*{14.}
Feminister er desperate etter å bevise at kvinner er like sterke og kapable som
menn. Det er innlysende at de frykter muligheten for at kvinner \emph{ikke} er
like sterke og kapable som menn.

\section*{15.}
Venstrevridde har en tendens til å hate alt som fremstår som sterkt, godt og
suksessfullt. De hater Amerika, vestlig sivilisasjon, hvite menn og
rasjonalitet. Grunnene venstrevridde gir for å hate vesten osv., samsvarer
tydelig ikke med deres egentlige motiver. De \emph{sier} at de hater vesten
fordi den er krigersk, imperialistisk, kjønnsdiskriminerende, etnosentrisk
osv., men når de samme fenomenene skjer i sosialistiske land eller i primitive
kulturer, finner den venstrevridde unnskyldninger for dem. I beste fall vil
vedkommende \emph{motvillig} si at de eksisterer, mens han i vestens tilfeller
vil \emph{entusiastisk} påpeke (og ofte overdrive) disse fenomenene. Derfor er
det klart at disse feilene ikke er den egentlige grunnen til at den
venstrevridde hater Amerika og vesten. Han hater Amerika og vesten på grunn av
deres styrke og suksess.

\section*{16.}
Ord som ``selvsikkerhet'', ``selvstendighet'', ``initiativ'',
``foretaksomhet'', ``optimisme'' osv.\ spiller en liten rolle i liberal og
venstrevridd språkbruk. Den venstrevridde er anti-individualistisk og
kollektivistisk. Han ønsker at samfunnet skal løse alles problemer,
tilfredsstille alles behov og ta vare på alle. Han er ikke typen som har en
indre selvsikkerhet når det kommer til hans evne til å løse sine egne problemer
eller tilfredsstille sine egne behov. Den venstrevridde misliker idéen om
konkurranse fordi han innerst inne føler seg som en taper.

\section*{17.}
Kunstformer som appellerer til moderne venstrelente intellektuelle har en
tendens til å fokusere på elendighet, tap og fortvilelse. Ellers har de en
ukontrollert tone som fjerner seg fra rasjonell kontroll som om det var ingen
håp for å kunne utrette noe ved hjelp av rasjonell kalkulering, og alt som var
igjen var å overgi seg til øyeblikkets følelser.

\section*{18.}
Moderne venstrelente filosofer har en tendens til å forkaste fornuft, vitenskap
og objektiv virkelighet, og insistere på at alt er kulturelt relativt. Det er
sant at man kan stille seriøse spørsmål om grunnlaget til vitenskapelig
kunnskap og om hvordan, om i det hele tatt, idéen om objektiv virkelighet kan
defineres. Men det er åpenbart at moderne venstrelente filosofer ikke er
sindige logikere som systematisk analyserer kunnskapens grunnlag. De er dypt
emosjonelt investert i å angripe sannhet og virkelighet. De angriper disse
konseptene på grunn av deres egne psykologiske behov. For det første er deres
angrep et utløp for fiendtlighet, og i den grad det lykkes, tilfredsstiller det
ønsket om makt. Viktigere hater den venstrevridde vitenskap og rasjonalitet
fordi de kategoriserer visse synspunkter som sanne (altså suksessrik,
overlegen) og andre synspunkter som falske (altså mislykket, underlegen). Den
venstrevriddes følelser av underlegenhet går så dypt at han ikke tåler noen
form for kategorisering som sier at noe er suksessrikt eller overlegent og
andre ting er mislykket eller underlegent. Dette underligger også mange
venstrevriddes forkastelse av idéen om psykiske lidelser og nytteverdien til
IQ-tester. Venstrevridde har et fiendtligt forhold til genetiske
forklaringsmodeller når det kommer til menneskelige evner og adferd, og dette
er fordi slike forklaringer har en tendens til å få noen mennesker til å
fremstå som sterkere eller svakere enn andre. Venstrevridde foretrekker å gi
samfunnet æren eller skylden for et individs evner eller mangel på evner.
Om en person er ``underlegen'' er det derfor ikke hans feil, men heller
samfunnets feil, fordi han ikke ble oppdratt på en god måte.

\section*{19.}
Den venstrevridde er vanligvis ikke typen som blir påvirket av sine følelser av
underlegenhet i den grad at han blir en skrytepave, egoist, mobber,
selvhevdende eller et kynisk konkurransemenneske. Denne typen person har ikke
helt mistet troen på seg selv. Han har mangler når det kommer til hans følelse
av makt og verdighet, men han kan likevel se seg selv som en med kapasitet til
å være sterk og hans innsats for å gjøre seg selv sterk forårsaker hans
uønskelige oppførsel.\footnote{Vi hevder ikke at alle, eller engang de fleste,
mobbere lider av følelser av underlegenhet.} Men den venstrevridde har gått for
langt for dette. Hans følelser av underlegenhet sitter så dypt at han ikke kan
se på seg selv som individuelt sterk og verdifull. Derfor er den venstrevridde
kollektivistisk. Han kan bare føle seg sterk som en del av stor organisasjon
eller bevegelse som han identifiserer seg med.

\section*{20.}
Legg merke til de masochistiske tendensene til venstrevridd strategi.
Venstrevridde protesterer ved å legge seg ned foran kjøretøy, og de provoserer
politiet og rasister slik at de skal misbruke dem osv. Disse strategiene kan
ofte være effektive, men mange venstrevridde bruker dem ikke for å nå et
bestemt mål. De bruker dem fordi de foretrekker masochistisk strategi. Selvhat
er et venstrevridd trekk.

\section*{21.}
Venstrevridde hevder kanskje at deres aktivisme er motivert av medfølelse eller
moralske prinsipper, og det er riktig at moralske prinsipper spiller en rolle
for den oversosialiserte venstrevridde typen. Men medfølelse og moralske
prinsipper kan ikke være hovedmotivene for venstrevridd aktivisme. Fiendtlighet
er en for fremtredende del av venstrevridd oppførsel, og det samme er ønsket om
makt. Dessuten er mye venstrevridd oppførsel ikke rasjonelt kalkulert til å
være til fordel for de menneskene som venstrevridde hevder å hjelpe. Om noen
for eksempel mener at kvotering er bra for svarte mennesker, gir det mening å
kreve kvotering på en fiendtlig eller dogmatisk måte? Åpenbart ville det vært
mer effektivt å gå for en diplomatisk og fleksibel løsning som i det minste
ville gitt verbale og symbolske forsikringer til hvite mennesker som mener at
kvotering er diskriminerende mot dem. Men venstrevridde aktivister tar ikke en
slik strategi i bruk fordi den ikke ville tilfredsstilt deres emosjonelle
behov. Å hjelpe svarte mennesker er ikke deres egentlige mål. Raserelaterte
problemer fungerer heller som en unnskyldning for at de skal uttrykke sin egen
fiendtlighet og behov for makt. Med å gjøre dette skader de faktisk svarte
mennesker fordi aktivistenes fiendtlige innstilling overfor den hvite
majoriteten har en tendens til å forsterke rasehat.

\section*{22.}
Om samfunnet vårt ikke hadde noen sosiale problemer i det hele tatt, måtte de
venstrevridde \emph{diktet opp} problemer for å gi seg selv en unnskyldning for
å lage bråk.

\section*{23.}
Vi understreker at det foregående hevder ikke å være en nøyaktig beskrivelse av
alle som muligens kan defineres som venstrevridd. Det er bare en grov
indikasjon på en generell venstrevridd tendens.

\chapter{Oversosialisering}
\section*{24.}
Psykologer bruker begrepet ``sosialisering'' for å beskrive prosessen der barn
trenes opp til å tenke og oppføre seg på den måten som samfunnet krever. En
person vil sies å være bra sosialisert hvis han tror på og adlyder samfunnets
moralske kode og passer bra inn som et velfungerende samfunnsmedlem. Det virker
kanskje meningsløst å si at mange venstrevridde er oversosialiserte fordi den
venstrevridde blir sett på som en opprører. Uansett kan dette synspunktet
forsvares. Mange venstrevridde er ikke like store opprørere som de virker som.

\section*{25.}
Den moralske koden i vårt samfunn er så krevende at ingen kan tenke, føle og
handle på en helt moralsk måte. Vi skal for eksempel ikke hate noen, men nesten
alle hater noen på et eller annet tidspunkt, uavhengig av om de innrømmer dette
overfor seg selv eller ikke. Noen mennesker er så sosialiserte at deres forsøk
på å tenke, føle og handle moralsk, pålegger dem en stor byrde. For å unngå
skyldfølelse må de hele tiden bedra seg selv når det gjelder sine egne motiver,
og de må finne moralske forklaringer for følelser og handlinger som i
realiteten ikke har et moralsk opphav. Vi bruker begrepet ``oversosialisert''
for a beskrive slike mennesker.\footnote{Under viktoriatiden led mange
mennesker av psykologiske problemer som kom som et resultat av, eller forsøk på
undertrykkelse av seksuelle følelser. Freud skal visstnok ha basert sine
teorier på denne typen mennesker. I dag har sosialiseringens fokus gått fra sex
til aggresjon.}

\section*{26.}
Oversosialisering kan føre til lav selvtillit, en følelse av maktesløshet,
defaitisme, skyldfølelse osv. En av de viktigste måtene samfunnet vårt
sosialiserer barn er ved å få dem til å føle skam når de handler eller snakker
på en måte som går mot samfunnets forventninger. Om dette overdrives, eller om
et bestemt barn er spesielt tilbøyelig for slike følelser, ender han opp med å
skamme {\em seg}. I tillegg er den oversosialiserte personens tanker og
oppførsel mer begrenset av samfunnets forventninger enn en som er lett
sosialisert. Flertallet av mennesker er delaktige i betydelige mengder dårlig
oppførsel. De lyver, nasker, bryter trafikkregler, unnasluntrer på jobb, hater
noen, sier fornærmende ting eller bruker sleipe triks for å komme seg foran
andre. Den oversosialiserte personen kan i utgangspunktet ikke gjøre disse
tingene, og om han gjør dem, vil han kjenne på følelser av skam og selvhat. Den
oversosialiserte personen kan ikke uten skyldfølelse oppleve tanker eller
følelser som går mot den alment aksepterte moralen; han kan ikke tenke
``skitne'' tanker. Og sosialisering er ikke bare et spørsmål om moral; vi
sosialiseres til å tilpasse oss mange adferdsnormer som ikke alle går under
kategorien moral. Ergo blir den oversosialiserte personen holdt med et
psykologisk bånd og går gjennom livet gående på veier som samfunnet har laget
for ham. For mange oversosialiserte personer forårsaker dette en følelse av
begrensning og maktesløshet som kan være svært vanskelig. Vi foreslår at
oversosialisering er blant de mer seriøse grusomhetene mennesker kan gjøre mot
hverandre.

\section*{27.}
Vi argumenterer for at en veldig viktig og innflytelsesrik del av den moderne
venstresiden er oversosialisert og at deres oversosialisering er av stor
viktighet når det kommer til å bestemme retningen til moderne venstrevridd
ideologi. Venstrevridde av den oversosialiserte typen har en tendens til å være
intellektuelle eller medlemmer av øvre middelklasse. Legg merke til at
intellektuelle ved universiteter\footnote{Spesialister innenfor ingeniørfag
eller ``seriøs'' vitenskap er ikke nødvendigvis inkludert her.} er den mest
sosialiserte, og også den mest venstrevridde delen av samfunnet.

\section*{28.}
Den venstrevridde av den oversosialiserte typen prøver å fjerne seg fra det
psykologiske båndet og hevde sin automoni ved å gjøre opprør, men vanligvis er
han ikke sterk nok til å gjøre opprør mot samfunnets mest grunnleggende
verdier. Generelt sett er målene til dagens venstrevridde \emph{ikke} i konflikt
med den alment aksepterte moralen. Tvert i mot vil venstresiden ta et akseptert
moralsk prinsipp, internalisere det, og så kritisere storsamfunnet for å bryte
det prinsippet. Eksempler er raselikhet, likestilling mellom kjønnene, å hjelpe
fattige, fred i stedet for krig, ikkevold generelt, frihet for å uttrykke seg
slik man vil og vennlighet mot dyr. Mer grunnleggende individets plikt til å
tjene samfunnet, og samsfunnets plikt til å ta seg av individet. Alle disse
verdiene har vært dypt rotfestet i vårt samfunn (eller i det minste i middel og
øvre klasser\footnote{Det er mange individer i middel og øvre klasser som
motsetter seg noen av disse verdiene, men vanligvis er deres motstand mer eller
mindre skjult. Slik motstand kommer frem i massemediene i en veldig begrenset
grad. Hoveddelen av propaganda i vårt samfunn er til fordel for fastsatte
normer. Hovedgrunnen til at disse verdiene har blitt så å si de offisielle
verdiene i vårt samfunn, er fordi de er nyttige for det industrielle samfunn.
Vold er frarådet fordi det forstyrrer systemets funksjon. Rasisme er frarådet
fordi etniske konflikter også forstyrrer systemet, og diskriminering kaster
bort talentene til minoriteter som kunne vært nyttig for systemet. Fattigdom må
``kureres'' fordi underklassen forårsaker problemer for systemet og kontakt med
underklassen senker moralen til de andre klassene. Kvinner blir oppfordret til
å ha karrierer fordi deres talenter er nyttige for systemet, og viktigere fordi
ved å ha vanlige jobber blir de bedre integrert inn i systemet og knyttes
direkte til systemet fremfor familiene deres. Dette hjelper med å svekke
samholdet i familien. (Systemets ledere sier at de ønsker å styrke familien,
men det de egentlig mener er at de ønsker at famlien skal fungere som et
effektivt verktøy for å sosialisere barn etter systemets behov. Vi argumenterer
i avsnittene 51 og 52 for at systemet ikke har råd til å la familien eller
andre små sosiale grupper bli sterke eller selvstendige.)}) lenge. Disse
verdiene er eksplisitt eller implisitt uttrykt eller forutsett i mesteparten av
stoffet som presenteres til oss av mediene og utdanningsinstitusjoner.
Venstrevridde, spesielt de av den oversosialiserte typen, vil vanligvis ikke
gjøre opprør mot disse prinsippene, men heller rettferdiggjøre deres
fiendtlighet overfor samfunnet ved å hevde (med en viss grad av sannhet) at
samfunnet ikke lever opp til dem.

\section*{29.}
Her er en illustrasjon på hvordan den venstrevridde viser hvor tett han er
knyttet til samfunnets konvensjonelle normer imens han later som å gjøre opprør
mot dem. Mange venstrevridde kjemper for kvotering, å flytte svarte mennesker
inn i presisjefylte jobber, forbedret utdanning i svarte skoler og mer penger
for slike skoler; den svarte ``underklassens'' levemåte ser de på som en skam.
De ønsker å integrere den svarte mannen inn i systemet ved å gjøre han til en
forretningsmann, advokat eller en vitenskapsmann akkurat som hvite mennesker i
den øvre middelklasse. De venstrevridde vil svare med å si at det siste de
ønsker å gjøre den svarte mannen til en kopi av den hvite mannen, og at de
ønsker å bevare afroamerikansk kultur. Men hva består denne bevaringen av? Den
kan knapt bestå av mer enn å spise afroamerikansk mat, høre på afroamerikansk
musikk, gå med afroamerikanske klær og å gå til en afroamerikansk kirke eller
moské. Med andre ord kan det bare uttrykkes i overfladiske tilfeller. På alle
\emph{vesentlige} måter vil venstrevridde av den oversosialiserte typen gjøre
slik at den svarte mannen må tilpasse seg idealene til den hvite middelklassen.
De ønsker at han skal studere tekniske fag, bli en leder eller vitenskapsmann
og å bruke livet sitt på å klatre karrierestigen for å bevise at svarte
mennesker er like gode som hvite. De ønsker å gjøre svarte fedre
``ansvarlige'', svarte gjenger ikkevoldelige osv. Men dette er det
industrielt-teknologiske systemets nøyaktive verdier. Systemet kunne ikke brydd
seg mindre om hva slags musikk man lytter til, hva slags klær man går med eller
hvilken religion man tror på så lenge man går på skolen, har en respektabel
jobb, klatrer karrierestigen, er en ``ansvarlig'' forelder, er ikkevoldelig
osv. Effektivt sett vil den venstrevridde integrere den svarte mannen inn i
systemet og få han til å internalisere dets verdier, uavhengig av hvor mye han
måtte fornekte dette.

\section*{30.}
Vi hevder absolutt ikke at venstrevridde, selv av den oversosialiserte typen,
\emph{aldri} gjør opprør mot vårt samfunns grunnleggende verdier. Åpenbart gjør
de dette noen ganger. Noen oversosialiserte venstrevridde har gått så langt som
å gjøre opprør mot et av samfunnets viktigste prinsipper ved bruke vold. Ifølge
deres egne ord, er vold for dem en form for ``frigjøring''. Med andre ord, ved
å bruke vold bryter de gjennom de psykologiske begrensningene som er
programmert inn i dem. Fordi de er oversosialisert har disse begrensningene
vært mer omfattende for dem enn for andre; ergo deres ønske om å bryte ut av
dem. Men de vil vanligvis rettferdiggjøre sitt opprør med verdier som er
vanlige. Om de bruker vold vil de hevde at de kjemper mot rasisme og lignende.

\section*{31.}
Vi innser at mange innvendinger kan komme mot den foregående beskrivelsen av
venstrevridd psykologi. Den virkelige situasjonen er komplisert, og noe som kan
nærme seg en fullstendig forklaring ville brukt flere bøker selv om den
nødvendige dataen var tilgjengelig. Vi hevder bare å ha indikert på en veldig
grov måte de to viktigste tendensene i psykologien til moderne venstrevridd
ideologi.

\section*{32.}
Den venstrevriddes problemer er også en indikasjon på samfunnets problemer
helhetlig. Lav selvtillit, depressive tendenser, og defaitisme er ikke
begrenset til venstresiden. Selv om de er spesielt synlige på venstresiden, er
de utbredt i vårt samfunn. Og dagens samfunn prøver å sosialisere i en større
grad enn noe tidligere samfunn. Vi blir til og med fortalt av eksperter hvordan
vi skal spise, trene, ha sex, oppdra våre barn osv.

\chapter{Maktprosessen}
\section*{33.}
Mennesker har et behov (mest sannsynlig biologisk) for noe vi kaller
maktprosessen. Dette er nærmt beslektet ønsket om makt (som er bredt
anerkjent), men ikke helt det samme. Maktprosessen har fire elementer. De tre
tydeligste av disse kaller vi mål, innsats og måloppnåelse. (Alle trenger å ha
mål der oppnåelsen krever innsats, og trengs for å oppnå minst noen av hans
mål.) Det fjerde elementet er vanskeligere å definere og er kanskje ikke
nødvendig for alle. Vi kaller det autonomi og vi vil diskutere det senere
(avsnitt 42--44).

\section*{34.}
Tenk på en hypotetisk mann som kan få hva han vil bare ved å ønske det. En slik
mann har makt, men han vil utvikle seriøse mentale problemer. I begynnelsen vil
han ha mye moro, men snart vil han bli høyst lei og demoralisert. Etterhvert
kan han bli klinisk deprimert. Historien viser oss at aristokratier med mye
fritid har en tendens til å forfalle. Dette er ikke tilfelle i kjempende
aristokratier som må streve for å beholde sin makt. Men artistokratier som er
sikre og har mye fritid, og som ikke trenger å anstrenge seg blir vanligvis
kjedsom, hedonistisk og demoralisert, selv om de har makt. Dette viser at makt
ikke er nok. Man må ha mål som man kan bruke sin makt til å nå.

\section*{35.}
Alle har mål; om ingenting annet, å få tak i det nødvendige man trenger for å
leve: mat, vann, klær og husly (de to siste vil variere etter klima). Men
aristokraten med mye fritid får tak i disse tingene uten innsats. Ergo hans
kjedsomhet og demoralisering.

\section*{36.}
Å ikke nå viktige mål fører til død om målene er fysiske nødvendigheter, og til
frustrasjon om det å ikke oppnå målene er forenelig med overlevelse. Å ikke
oppnå sine mål gjennom livet fører til defaitisme, lav selvtillit eller
depresjon.

\section*{37.}
For å unngå seriøse psykologiske problemer, må et menneske derfor ha mål der
oppnåelsen krever innsats, og han må ha en rimelig grad av suksess i oppnåelsen
av sine mål.

\chapter{Erstatningsaktiviteter}
\section*{38.}
Men ikke alle aristokrater med mye fritid blir lei og demoralisert. Keiser
Hirohito, for eksempel, henga seg selv til marinebiologi, et felt der han ble
anerkjent, i stedet for å synke ned til dekadent hedonisme. Når folk ikke
trenger å anstrenge seg for å tilfredsstille sine fysiske behov, setter de ofte
opp kunstige mål for seg selv. I mange tilfeller vil de jage disse målene med
den samme energien og emosjonelle enasjementet som de ville gjort om det var
snakk om fysiske nødvendigheter. Derfor hadde artistokratene i det romerske
imperiet sine litterære påskudd; mange europeiske aristokrater brukte svært mye
tid og energi på jakt, selv om de med sikkerhet ikke trengte kjøttet; andre
aristokratier har konkurrert om status gjennom kunstferdig fremvisning av
rikdom; og noen få aristokrater, som Hirohito, gikk til vitenskapen.

\section*{39.}
Vi bruker begrepet ``erstatningsaktivitet'' for å beskrive en aktivitet som
gjøres for å nå et abstrakt mål som folk setter for seg selv bare for å ha et
mål å jobbe mot, eller la oss si, bare for ``tilfredsstillelsen'' de får ved å
jage det målet. Her er en tommelfingerregel for å identifisere 
erstatningsaktiviteter. Gitt en person som bruker mye tid og 
energi for å nå mål X, spør deg selv dette: om han måtte 
bruke mesteparten av tiden sin på å tilfredsstille sine biologiske behov, 
og om den innsatsen krevde at han brukte sine fysiske og mentale evner 
på en variert og interessant måte, ville han følt seg berøvet fordi 
han ikke nådde mål X?\ Om svaret er nei er personens jag etter å nå
mål X en erstatningsaktivitet. Hirohito sine studier innenfor 
marinebiologi var tydelig et eksempel på en erstatningsaktivitet 
fordi det er ganske sikkert at om Hirohito måtte bruke tiden sin 
på å jobbe med interessante ikke-vitenskapelige oppgaver for å oppnå 
livets nødvendigheter, ville han ikke følt seg berøvet fordi han ikke 
visste alt om anatomien og livssyklusen til sjødyr. På den andre siden 
er jaget etter sex og kjærlighet (for eksempel) ikke en erstatningsaktivitet 
fordi de fleste mennesker, selv om deres eksistens var ellers tilfredsstillende, 
ville følt seg berøvet om de gikk gjennom livet uten å ha et forhold til et 
medlem av det motsatte kjønn. (Men jag etter en overdreven mengde med sex kan 
være en erstatningsaktivitet.)

\section*{40.}
I det moderne industrielle samfunn er svært lite innsats krevd for å
tilfredsstille sine fysiske behov. Det er nok å gå gjennom et
arbeidstreningsprogram for å få en eller annen ubetydelig teknisk ferdighet,
for så å komme tidsnok til jobb og anstrenge seg i en liten grad for å beholde
jobben. De eneste kravene er en moderat mengde intelligens og viktigst av alt,
{\em lydighet}. Om man har disse, vil samfunnet ta vare på deg fra vugge til
grav. (Ja, det er en underklasse som ikke kan ta de fysiske nødvendighetene for
gitt, men vi snakker her om det generelle samfunnet.) Derfor er det ikke
overraskende at det moderne samfunn er fullt av erstatningsaktiviteter. Disse
inkluderer vitenskelig arbeid, atletisk prestasjon, humanitært arbeid,
kunstnerisk og litterær skapelse, å klatre karrierestigen, anskaffelse av
penger og materielle goder som går mye lengre enn det som er nødvendig for å få
mer fysisk tilfredsstillelse og aktivisme som går på problemer som ikke er
viktig for aktivisten personlig (som er tilfelle når det kommer til hvite
aktivister som jobber for rettighetene til ikkehvite minoriteter). Disse er
ikke alltid \emph{rene} erstatningsaktiviteter fordi for mange kan de være
motivert delvis av andre behov enn å ha et mål å jobbe mot. Vitenskapelig
arbeid kan være motivert delvis av et ønske om prestisje, kunstnerisk skapelse
av et behov for å uttrykke følelser og militant aktivisme av fiendtlighet. Men
for de fleste som jager dem, er disse aktivitetene for det meste
erstatningsaktiviteter. For eksempel vil nok de fleste vitenskapsfolk si seg
enig i at ``tilfredsstillelsen'' de får av jobben er viktigere enn pengene og
presisjen de får.

\section*{41.}
For mange, om ikke alle, er erstatningsaktiviteter mindre tilfredsstillende enn
jaget etter ekte mål (mål som folk ønsker å nå selv om deres behov for
maktprosessen allerede var oppnådd). En indikasjon på dette er faktumet (i
mange eller de fleste tilfeller) at folk som er dypt engasjert i
erstatningsaktiviteter aldri er tilfredsstilt, aldri hviler. Derfor vil
pengesamleren alltid prøve å få mer penger. Vitenskapsmannen vil knapt løse ett
problem før han går over til det neste. Langdistanseløperen presser seg selv
hele tiden til å løpe lengre og fortere. Mange som gjør disse
erstatningsaktivitetene vil si at de får mye mer tilfredsstillelse fra disse
aktivitetene enn de får fra den ``kjedelige'' jobben med å tilfredsstille sine
biologiske behov, men det er fordi at i vårt samfunn har innsatsen som kreves
for å tilfredsstille sine biologiske behov blitt redusert til noe ubetydelig.
Enda viktigere tilfredsstiller folk i vårt samfunn ikke sine biologiske behov
{\em selvstendig}, men heller ved å fungere som deler i en enorm sosial maskin.
Derimot har folk generelt mye autonomi når det kommer til å delta i
erstatningsaktiviteter.

\chapter{Autonomi}
\section*{42.}
Autonomi som en del av maktprosessen er kanskje ikke nødvendig for hvert
individ. Men de fleste mennesker trenger en større eller mindre grad av
autonomi når de jobber mot sine mål. Deres innsats må gjøres på deres eget
initiativ og må være under deres ledelse og kontroll. Likevel må ikke de fleste
bruke initiativet, ledelsen og kontrollen som enkeltindivider. Det er vanligvis
nok å handle som en del av en \emph{liten} gruppe. Derfor, om et halvt dusin med
mennesker diskuterer et mål mellom seg og går inn for å nå det målet med
suksess, vil deres behov for maktprosessen være tilfredsstilt. Men om de jobber
under rigide ordre ovenfra som ikke gir dem noe rom for selvstendige
avgjørelser og initiativ, vil deres behov for maktprosessen ikke være
tilfredsstilt. Det samme er sant når avgjørelser tas på en kollektiv basis om
gruppen som tar den kollektive avgjørelsen er så stor at rollen til hvert
individ er ubetydelig.\footnote{Det kan kanskje argumenteres for at flertallet
av mennesker ikke ønsker å ta sine egne valg, men heller la ledere tenke for
dem. Det er en grad av sannhet i dette. Folk liker å ta sine egne valg i små
saker, men å ta valg når det kommer til vanskelige og fundamentale spørsmål
krever å møte psykologiske konflikter, og de fleste hater psykologiske
konflikter. Ergo har de en tendens til å lene seg på andre når det kommer til å
ta vanskelige valg. Men dette betyr ikke at de liker når andre tar valg uten at
de har noen innflytelse på de valgene. Flertallet av mennesker er naturlige
følgere, ikke ledere, men de liker å ha direkte personlig tilgang til sine
ledere og de ønsker å ha muligheten til å påvirke lederne og delta i en viss
grad selv når det kommer til å ta vanskelige valg. De trenger autonomi i minst
den grad.}

\section*{43.}
Det er sant at noen individer virker å ha lite behov for autonomi. Enten er
ønsket deres om makt svakt, eller så tilfredsstiller de det med å identifisere
seg med en eller annen mektig organisasjon som de tilhører. Og i tillegg har
man også de tanketomme, animalistiske typene som virker å være tilfredsstilt av
et rent fysisk konsept av makt (den gode soldaten, som får sin følelse av makt
ved å utvikle kampegenskaper som han med glede bruker med blind lydighet
overfor sine ledere).

\section*{44.}
Men for de fleste er det gjennom maktprosessen --- å ha et mål, gjøre en
\emph{selvstendig} innsats, og nå det målet --- at selvtillit, selvsikkerhet og
en
følelse av makt oppnås. Når noen ikke har en tilstrekkelig mulighet til å gå
gjennom maktprosessen er konsekvensene (avhengig av individet og måten
maktprosessen forstyrres) kjedsomhet, demoralisering, lav selvtillit, følelser
av underlegenhet, defaitisme, depresjon, angst, skyldfølelse, frustrasjon,
fiendtlighet, vold i hjemmet, umettelig hedonisme, uvanlig seksuell oppførsel,
søvnplager, spiseforstyrrelser osv.\footnote{Noen av symptomene oppført er
like de som vises av dyr i bur. For å forklare hvordan disse symptomene kommer
fra berøvelse når det kommer til maktprosessen: en forståelse av menneskelig
natur basert på sunn fornuft sier til oss at en mangel på mål som krever
innsats fører til kjedsomhet og den kjedsomheten vil før eller siden ofte føre
til depresjon. Å ikke klare å nå mål fører til frustrasjon og lavere
selvtillit. Frustrasjon fører til sinne, sinne til aggresjon (ofte i form av
vold i hjemmet). Det har blitt vist at en langvarig frustrasjon ofte fører til
depresjon og depresjonen forårsaker skyldfølelse, søvnplager,
spiseforstyrrelser og dårlige følelser om seg selv. De som blir depressive vil
søker nytelse som en motgift; ergo umettelig hedonisme og en overdreven mengde
sex med perversitet som en måte å få nye spenninger. Kjedsomhet har også en
tendens til å forårsake overdrevne søk etter nytelse fordi at uten andre mål
vil folk ofte bruke nytelse som et mål.}

\chapter{Årsaker til samfunnsmessige problemer}
\section*{45.}
Hvilken som helst av de tidligere beskrevne symptomene kan forekomme i alle
samfunn, men i det moderne industrielle samfunnn vil de eksistere i en enorm
skala. Vi er ikke de første som nevner at verden i dag virker å bli gal. Denne
typen ting er ikke normale for menneskelige samfunn. Det er gode grunner til å
tro at den primitive mannen var mindre stresset og frustrert, og mer tilfreds
med livet enn den moderne mannen er. Det er sant at det ikke bare var gull og
grønne skoger i primitive samfunn. Mishandling av kvinner var vanlig i den
australske urbefolkningen og transseksualitet var ganske vanlig blant noen av
de amerikanske indianerstammene. Men det virker som at på et \emph{generelt
grunnlag} var de problemene som vi har oppført mye mindre vanlig i primitive
samfunn enn de er i det moderne samfunn.

\section*{46.}
Vi tillegger de sosiale og psykologiske problemene i det moderne samfunn til
det faktum at samfunnet krever at mennesker lever under forhold som er radikalt
forskjellig sammenlignet med de som menneskene utviklet seg under og at
mennesker oppfører seg på måter som strider med oppførselsmønstrene som
mennesker utviklet når de levde under tidligere forhold. Det er klart fra det
vi allerede har skrevet at vi ser på en mangel på muligheten til å skikkelig
oppleve maktprosessen som den viktigste av de unormale forholdene som det
moderne samfunn tvinger folk til å leve under. Men det er ikke den eneste. Før
vi snakker om forstyrrelse av maktprosessen som en kilde til sosiale problemer
må vi snakke om noen av de andre kildene. 

\section*{47.}
Blant de unormale forholdene som er til stede i det moderne industrielle
samfunnet er overdreven befolkningstetthet, isolering av mennesket fra naturen,
overdreven hurtighet når det kommer til sosiale endringer og ødeleggelsen av
naturlige småskala samfunn slik som den utvidede familien, landsbyen eller
stammen. 

\section*{48.}
Det er velkjent at trengsel øker stress og aggresjon. Graden av trengsel som
eksisterer i dag og isoleringen av mennesket fra naturen er konsekvenser av
teknologisk fremgang. Alle førindustrielle samfunn var i all hovedsak rurale.
Den industrielle revolusjon økte størrelsen massivt på byer og andelen av
befolkningen som bor i dem, og moderne landbruksteknologi har gjort det mulig
for jorden å bære en mye tettere befolkning enn noen gang tidligere. (Teknologi
forverrer også effektene av trengsel fordi den legger mer makt til å forstyrre
i folks hender. For eksempel en rekke støylagende enheter: gressklippere,
radioer, motorsykler osv. Om bruken av disse enhetene er uten begrensninger,
vil folk som ønsker fred og ro bli frustrert av lyden. Om bruken begrenses, vil
folk som bruker enhetene bli frustrert av reguleringene. Men om disse maskinene
aldri hadde blitt oppfunnet i utgangspunktet, ville de ikke forårsaket konflikt
og frustrasjon.)

\section*{49.}
For primitive samfunn ga den naturlige verden (som vanligvis bare endrer seg
gradvis) et stabilt rammeverk og derfor en følelse av sikkerhet. I den moderne
verden er det menneskelige samfunn som dominerer naturen og ikke omvendt, og
det moderne samfunn endrer seg veldig fort i takt med teknologiske endringer.
Ergo er det ikke noe stabilt rammeverk.

\section*{50.}
De konservative er tosker: de syter over forfallet til tradisjonelle verdier,
men samtidig støtter de entusiastisk teknologisk fremgang og økonomisk vekst.
Tilsynelatende påfaller det dem aldri at du ikke kan gjøre raske og drastiske
endringer i teknologien og økonomien i et samfunn uten å forårsake raske
endringer i andre samfunnsaspekter også, og at slike raske endringer uunngåelig
bryter ned tradisjonelle verdier.

\section*{51.}
Nedbrytningen av tradisjonelle verdier medfører til en viss grad nedbrytningen
av båndene som holder tradisjonelle småskala sosiale grupper sammen.
Oppløsningen av av småskala sosiale grupper promoteres også av det faktum at
moderne forhold ofte krever eller frister individer til å flytte til nye
steder, som skiller dem fra sine lokalsamfunn. Videre \emph{må} et teknologisk
samfunn svekke familiebånd og lokalsamfunn om det skal fungere effektivt. I det
moderne samfunn må et individ først være lojal overfor systemet og bare
sekundært til lokalsamfunnet fordi om de interne lojalitetene i småskala
samfunn var sterkere enn lojaliteten til systemet, ville slike samfunn gå etter
sin egen fordel på bekostning av systemet. 

\section*{52.}
Se for deg at en offentlig embetsmann eller høytstående i et privat selskap
ansetter sitt søskenbarn, sin venn eller medtroende fremfor den som er best
kvalifisert til jobben. Han har gjort en handling som viser at han setter
personlig lojalitet foran lojaliteten til systemet, og det er ``nepotisme''
eller ``diskriminering'', som begge er store synder i det moderne samfunn.
Samfunn som ellers hadde vært industrielle og som har gjort en dårlig jobb når
det kommer til å få folk til å sette lojalitet overfor systemet over personlig
lojalitet, er vanligvis veldig ineffektive. (Se på Latin-Amerika.) Derfor kan
et avansert industrielt samfunn bare tolerere de småskala samfunnene som er
kastrerte, tamme og gjort om til systemets verktøy.\footnote{Et delvis unntak
kan gjøres for et par passive og innadvendte grupper slik som amish-folket, som
påvirker storsamfunnet i en svært liten grad. Bortsett fra disse, eksisterer
noen få genuine småskala samfunn i dagens USA.\ Ungdomsgjenger og ``kulter'' er
to eksempler. Alle ser på dem som farlige, og det er de, fordi at medlemmene i
disse gruppene er hovedsakelig lojale overfor hverandre og ikke systemet, så
systemet kan ikke kontrollere dem. Sigøynerne er et annet eksempel. Sigøynerne
kommer seg ofte unna med stjeling og svindel fordi deres lojalitet er slik at
de alltid kan få andre sigøynere til å avlegge vitnesbyrd som ``beviser'' deres
uskyld. Systemet ville åpenbart fått mange problemer om for mange mennesker
tilhørte slike grupper. Noen av tenkerne tidlig på 1900-tallet i Kina som var
bekymret for moderniseringen i Kina så nødvendigheten av å bryte ned småskala
sosiale grupper slik som familien: '' (Ifølge Sun Yat-sen) trengte kineserne en
ny bølge av patriotisme, som ville overføre lojalitet fra familien til
systemet\ldots (Ifølge Li Huang) måtte tradisjonelle koblinger, særlig familien,
forlates om nasjonalisme skulle utvikle seg i Kina. '' (Chester C. Tan,
``Chinese Political Thought in the Twentieth Century, '' side 125 og 297.)}

\section*{53.}
Trengsel, raske endringer og nedbrytningen av samfunn har bredt blitt sett på
som kilder til samfunnsmessige problemer. Men vi tror ikke at de er nok til å
forklare omfanget av problemene vi ser i dag. 

\section*{54.}
Noen få førindustrielle samfunn var veldig store og overfylte, men likevel
virker det ikke som at deres innbyggere led av psykologiske problemer i samme
grad som det moderne mennesket. I dagens USA er det fremdeles rurale områder
som ikke er overfylte, og vi finner de samme problemene der som i urbane
områder, selv om problemene virker å være mindre akutt i rurale områder. Ergo
virker det ikke som at trengsel er den avgjørende faktoren. 

\section*{55.}
På den voksende grensen av den amerikanske periferien på 1800-tallet, brøt
befolkningens mobilitet ned storfamilier og småskala sosiale grupper minst til
den samme grad som disse brytes ned i dag. Faktisk valgte mange kjernefamilier
å leve i en slik isolasjon der de ikke hadde noen naboer på mange kilometer
slik at de ikke tilhørte noe samfunn i det hele tatt, men likevel virker det
ikke som at de utviklet problemer som et resultat.

\section*{56.}
Videre var endringer i den amerikanske periferien veldig hurtige og dype. En
mann var kanskje født og oppvokst i en tømmerhytte, utenfor lovens rekkevidde
og ernært hovedsakelig på viltkjøtt, og innen han ble gammel hadde han kanskje
en vanlig jobb og levde i et strukturert samfunn med effektiv håndhevelse av
loven. Dette var en dypere endring enn det som typisk er vanlig i livet til et
moderne individ, men likevel virker det ikke som at det førte til psykologiske
problemer. Faktisk hadde Amerika på 1800-tallet en optimistisk og selvsikker
tone. Veldig ulikt dagens samfunn.\footnote{Ja, vi er klar over at Amerika på
1800-tallet hadde sine problemer, og virkelig seriøse problemer, men for å være
kortfattet må vi forklare oss på en forenklet måte.}

\section*{57.}
Vi argumenterer for at den moderne mannen har følelsen av (i stor grad
rettferdiggjort) at endringer blir presset på han, men på 1800-tallet hadde
menn i periferien (også i stor grad rettferdiggjort) følelsen av at han skapte
endringer selv av egen vilje. Ergo ville en pioner bosette seg på en jordflekk
som han valgte selv og gjøre det om til en gård med egen innsats. På den tiden
hadde kanskje et helt lokalsamfunn bare noen hundre innbyggere og var mye mer
isolert og selvstendig enn et moderne lokalsamfunn er. Derfor deltok den
innovative bonden som et medlem av en relativt liten gruppe i skapelsen av et
nytt og strukturert samfunn. En kan kanskje stille spørsmål om skapelsen av
dette samfunnet var en forbedring, men uansett tilfredsstilte det pionerens
behov for maktprosessen.

\section*{58.}
Det ville vært mulig å gi andre eksempler på samfunn der det har vært hurtige
endringer og/eller mangel på tette samfunnsbånd uten den typen massive
oppførselsmessige avvik som vi ser i dagens industrielle samfunn. Vi hevder at
den viktigste årsaken til sosiale og psykologiske problemer i det moderne
samfunn er det faktum at folk har en utilstrekkelig mulighet til å gå gjennom
maktprosessen på en normal måte. Vi mener ikke å si at det moderne samfunn er
det eneste samfunnet der maktprosessen har blitt forstyrret. Mest sannsynlig
har de fleste, om ikke alle, siviliserte samfunn grepet inn i maktprosessen i
større eller mindre grad. Men i det moderne industrielle samfunn har problemet
blitt spesielt akutt. Venstrevridd ideologi, i det minste i sin nylige form
(midten til slutten av 1900-tallet), er til dels et symptom på berøvelse med
tanke på maktprosessen.

\chapter{Forstyrrelse av maktprosessen i det moderne samfunn}
\section*{59.}
Vi deler menneskelige behov inn i tre grupper: (1) de behovene som kan
tilfredsstilles med minimal innsats; (2) de som kan tilfredsstilles men bare
med stor innsats; (3) de som ikke tilstrekkelig kan tilfredsstilles uavhengig
av hvor mye innsats man legger ned. Maktprosessen er prosessen der man
tilfredsstiller behovene til den andre gruppen. Jo flere behov det er i den
tredje gruppen, jo mer frustrasjon, sinne, senere defaitisme, depresjon osv.
er det.

\section*{60.}
I det moderne industrielle samfunn har menneskelige behov en tendens til å bli
dyttet inn i den første og tredje gruppen, og den andre gruppen har en tendens
til å bestå i økende grad av kunstig skapte behov. 

\section*{61.}
I primitive samfunn, havner fysiske behov som regel i den andre gruppen: De kan
oppnås, men bare om det legges ned seriøs innsats. Men moderne samfunn har en
tendens til å garantere fysiske behov til alle\footnote{Vi legger
``underklassen'' til side. Vi snakker om den generelle befolkningen.} i bytte
mot bare en minimal innsats. Ergo dyttes fysiske behov inn i den andre gruppen.
(Det er kanskje uenigheter om hvorvidt innsatsen som kreves for å ha en jobb er
``minimal'', men som regel er kravene som stilles til jobber på lavere til
middels nivå bare at du er LYDIG.\ Du sitter eller står der du blir fortalt at
du skal sitte eller stå og du gjør ting på den måten du blir fortalt at det
skal gjøres. Det er sjeldent at du må anstrenge deg skikkelig, og uansett har
du omtrent ikke autonomi i jobben din slik at behovet for maktprosessen ikke
blir tilfredsstilt.)

\section*{62.}
Sosiale behov, slik som sex, kjærlighet og status, vil ofte bli i den andre
gruppen i det moderne samfunn, avhengig av individets situasjon.\footnote{Noen
samfunnsvitere, lærere, profesjonelle innen ``mental helse'' og lignende prøver
sitt beste på å dytte de sosiale motivene inn i den første gruppen ved å prøve å
se til at alle har et tilfredsstillende sosialt liv.} Men unntatt
folk som har et særlig stort driv for å oppnå status, er innsatsen som kreves
for å realisere de sosiale motivene utilstrekkelig for å tilfredsstille behovet
for maktprosessen i stor nok grad.

\section*{63.}
Så visse kunstige behov har blitt skapt som faller under den andre gruppen, og
tjener dermed behovet for maktprosessen. Annonsering- og
markedsføringsstrategier har blitt utviklet på en slik måte at folk føler de
trenger ting som deres besteforeldre aldri ønsket eller drømte om. Det krever
seriøs innsats for å tjene nok penger til å tilfredsstille disse kunstige
behovene, ergo faller de inn i den andre gruppen (Men se på avsnittene 80--82.).
Den moderne mannen må i stor grad tilfredsstille sitt behov for maktprosessen
gjennom jag etter kunstige behov\footnote{Er motivasjonen for endeløs materiell
anskaffelse virkelig en kunstig skapelse av annonserings- og
markedsføringsindustrien? Det er utvilsomt at det ikke noe underliggende
menneskelig driv for materiell anskaffelse. Det har vært mange kulturer der folk
har ønsket lite materiell velstand utover det som var nødvendig for å
tilfredsstille deres grunnleggende fysiske behov (Australske aboriginere, tradisjonell meksikansk
bondekultur, noen afrikanske kulturer). På den andre siden har det også vært
mange før-industrielle kulturer der materiell anskaffelse har spilt en viktig
rolle. Så vi kan ikke hevde at dagens anskaffelsorienterte kultur er utelukkende
en skapelse av annonserings- og markedsføringsindustrien. Men det er klart at
annonserings- og markedsføringsindustrien har spilt en viktig rolle i å skape
den kulturen. De store selskapene som bruker millioner på reklamer ville ikke
brukt den summen med penger med mindre de hadde gode beviser på at de ville få
det tilbake i form av økt salg. Et medlem av frihetsklubben møtte en salgssjef
for noen år siden som var ærlig nok til å fortelle han, ``Vår jobb er å få folk
til å kjøpe ting de ikke vil ha og ikke trenger''. Han forklarte så hvordan en
utrent nybegynner kunne gi folk fakta om produktet og ikke selge noe i det hele
tatt, men en trent og profesjonell selger ville selge mange produkter til de
samme kundene. Dette viser at folk blir manipulert til å kjøpe ting de egentlig
ikke vil ha.}, og gjennom erstatningsaktiviteter.

\section*{64.}
Det virker som at for mange folk, kanskje flertallet, at disse kunstige formene
for maktprosessen er utilstrekkelig. Et tema som dukker opp flere ganger i
tekstene til samfunnskritikerne fra den andre halvdelen av det 20.århundre er
følelsen av meningsløshet som påvirker mange mennesker i det moderne samfunn.
(Denne meningsløsheten blir ofte kalt andre ting slik som ``normløshet'' eller
``tomhet''). Vi antyder at den såkalte ``identitetskrisen'' egentlig er søken
etter mening, ofte for å forplikte seg til en passelig erstatningsaktivitet. Det
kan godt være at eksistensialismen i stor grad er et svar på meningsløsheten i
det moderne liv.\footnote{Meningsløshetens problem virker å ha blitt mindre
seriøst i løpet av de siste 15 årene eller noe slikt fordi at folk føler mindre
sikkerhet både økonomisk og i forhold til fysiske behov, sammenlignet med
tidligere, og behovet for sikkerhet gir dem et mål. Men meningsløshet har blitt
erstattet med frustrasjon over vanskeligheten med å skaffe sikkerhet. Vi legger
vekt på meningsløshetens problem fordi de liberale og venstrevridde ønsker å
løse våre samfunnsmessige problemer med å la samfunnet garantere alles
sikkerhet, men om det kan gjøres vil det bare bringe tilbake meningsløshetens
problem. Det virkelige problemet er ikke om samfunnet forsørger på en god eller
dårlig måte i forhold til sikkerhet, men heller at folk er avhengig av systemet
i stedet for å ha det i sine egne hender. Dette, forresten, er en del av grunnen
til at folk blir oppbrakt når det kommer til retten til å bære våpen; besittelse
av våpen legger det aspektet av sikkerhet i deres egne hender.} Søken etter
``tilfredsstillelse'' er veldig utbredt i det moderne samfunn. Men vi tenker at for de
fleste som utfører en aktivitet med mål om tilfredsstillelse (som er en
erstatningsaktivitet), gir dette ikke tilstrekkelig tilfredsstillelse. Med andre
ord oppfyller det ikke behovet for maktprosessen. (Se avsnitt 41.) Det
behovet kan bare bli fullstendig oppfylt ved å gjøre aktiviteter som har
et eksternt mål, slik som fysiske behov, sex, kjærlighet, status, hevn osv.

\section*{65.}
Der mål blir jaget med å tjene penger, klatre karrierestigen eller fungere som
en del av systemet på en annen måte, vil de fleste folk ikke være i en posisjon
der de kan følge sine mål \emph{selvstendig}. De fleste arbeidere er ansatt av
noen og, som vi poengterte i avsnitt 61, må bruke sine dager på å gjøre det de
blir fortalt og på den måten de blir fortalt. Selv de fleste folk som er
selvstendig næringsdrivende har begrenset med autonomi. Det er en gjengående
klage fra eiere av små bedrifter og entreprenører at hendene deres blir bundet
av overdreven statlig regulering. Noen av disse reguleringene er utvilsomt
unødvendige, men for det meste er statlige reguleringer nødvendige og
uunngåelige deler av vårt ekstremt komplekse samfunn. En stor del av små
bedrifter i dag er en del av et franchise system. Det ble rapportert i Wall
Street Journal for noen år siden at mange av franchiseselskapene krever at
søkere som ønsker å starte en franchise skal ta en personlighetstest som er
skapt slik at den skal \emph{eksludere} de som er kreative og tar initiativ.
Dette er fordi at disse personene ikke er lydige nok til å gå i takt med
franchise systemet. Dette ekskluderer mange av de folkene som mest trenger
autonomi fra små bedrifter.

\chapter{Hvordan noen mennesker tilpasser seg}

\chapter{Forskeres motiver}

\chapter{Frihetens natur}

\chapter{Noen historiske prinsipper}

\chapter{Det industrielt-teknologiske samfunn kan ikke reformeres}

\chapter{Begrensning av frihet er uunngåelig i det industrielle samfunn}

\chapter{De `negative' aspektene av teknologi kan ikke adskilles fra de `positive'}

\chapter{Teknologi er en mektigere samfunnsmessig kraft enn ønsket om frihet}

\chapter{Enklere samfunnsmessige problemer har vist seg å være uoverkommelige}

\chapter{Revolusjon er enklere enn reform}

\chapter{Kontroll av menneskelig adferd}

\chapter{Menneskeheten ved et veiskille}

\chapter{Menneskelig lidelse}

\chapter{Fremtiden}

\chapter{Strategi}

\chapter{To former for teknologi}

\chapter{Faren for venstrevridd ideologi}

\chapter{Sluttord}

\end{document}
