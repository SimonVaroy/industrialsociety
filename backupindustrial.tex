\documentclass[oneside]{book}
\usepackage[norsk]{babel}

\title{Det industrielle samfunn og dets fremtid}
\author{Theodore Kaczynski//Oversatt av Simon Nikolai Varøy}
\date{19. september 1995}

\begin{document}
\maketitle
\tableofcontents

\chapter{Introduksjon}
\section*{1.}
Den industrielle revolusjon og dens konsekvenser har vært en katastrofe for menneskeheten. Den har i høy grad økt forventet levealder for de av oss som bor i ''utviklede'' land, men den har destabilisert samfunnet, gjort livet utilfredsstillende, utsatt mennesker for uverdigheter, ført til omfattende psykologisk lidelse (i den tredje verden også fysisk lidelse) og påført naturen stor skade. Den kontinuerlige utviklingen av teknologi vil forverre situasjonen. Den vil sannelig utsette mennesker for større uverdigheter og påføre naturen større skade. Den vil også sannsynligvis føre til mer sosial uro, psykologisk lidelse og muligens mer fysisk lidelse selv i ''utviklede'' land.

\section*{2.}
Det industrielt-teknologiske systemet kan overleve, men det kan også kollapse. Om det overlever, er det en mulighet for at det vil oppnå en lav grad av fysisk og psykologisk lidelse. Samtidig vil dette bare kunne skje etter en lang og smertefull tidsperiode, og kostnaden vil være at mennesker og mange andre levende organismer reduseres permanent til fabrikkerte produkter og brikker i et større spill. Om systemet overlever, vil konsekvensene i tillegg være uunngåelige: Det er ingen måte å reformere eller modifisere systemet slik at man forhindrer det fra å frata folk verdighet og autonomi.

\section*{3.}
Om systemet kollapser vil konsekvensene fortsatt være svært smertefulle. Men jo større systemet blir, jo verre vil konsekvensene av kollapsen bli. Så om systemet skal kollapse, er det bedre at det skjer tidlig enn sent.

\section*{4.}
Vi tar derfor til orde for en revolusjon mot det industrielle systemet. Denne revolusjonen kan ta i bruk voldsmetoder eller ikke og den kan være spontan eller gå gradvis over noen tiår. Vi kan ikke forutse noe av dette, men vi legger frem på en veldig generell måte hvilke metoder de som hater det industrielle systemet bør ta i bruk for å legge til rette for en revolusjon mot denne typen samfunnsstruktur. Dette vil ikke være en {\em politisk} revolusjon. Dens mål vil ikke være å styrte myndigheter, men heller det økonomiske og politiske grunnlaget for dagens samfunn.

\section*{5.}
I denne artikkelen setter vi bare lys på noen av de negative utviklingene som har kommet fra det industrielt-teknologiske systemet. Andre slike utviklinger nevner vi bare kort eller ignorerer fullstendig. Dette betyr ikke at vi ser på disse andre utviklingene som uviktige. Av praktiske hensyn må vi avgrense diskusjonen vår til områder som har fått for lite offentlig oppmerksomhet eller der vi har noe nytt å si.

\chapter{Psykologien bak moderne venstrevridd ideologi}
\section*{6.}
Nesten alle vil si seg enig i at vi lever i et svært problemfullt samfunn. En av de mest omfattende manifestasjonene av denne galskapen i vår verden er venstrevridd ideologi, så en diskusjon om psykologien bak denne ideologien kan fungere som en introduksjon til diskusjonen om problemene med det moderne samfunn generelt.

\section*{7.}
Men hva er venstrevridd ideologi? Under den første halvdelen av 1900-tallet kunne man praktisk talt identifisert venstrevridd ideologi som sosialisme. I dag er bevegelsen fragmentert og det er ikke klart hvem som kan defineres som venstrevridd. Når vi snakker om venstrevridde i denne artikkelen tenker vi for det meste på sosialister, kollektivister, ''politisk korrekte'' typer, feminister, homo- og handikap aktivister, dyrerettsaktivister og lignende. Men ikke alle som er assosiert med en av disse bevegelsene er venstrevridd. Det vi sikter til når vi diskuterer venstrevridd ideologi er ikke i noen stor grad en bevegelse eller ideologi, men heller en psykologisk type eller en samling av relaterte typer. Derfor vil det vi mener med ''venstrevridd ideologi'' komme frem tydeligere i løpet av vår diskusjon om venstrevridd psykologi. (Se avsnitt 227-30.)

\section*{8.}
Likevel vil vår forståelse av venstrevridd ideologi forbli en god del mindre klar enn det vi skulle ønske, men det virker ikke som at det er noen måte å fikse dette. Alt vi prøver å gjøre her er å indikere på en grov og omtrentlig måte de to psykologiske tendensene som vi mener er den største drivkraften bak moderne venstrevridd ideologi. Vi hevder på ingen måte at vi forteller {\em hele} sannheten om venstrevridd psykologi. Vår diskusjon er også bare ment å gjelde for moderne venstrevridd ideologi. Spørsmålet om i hvor stor grad diskusjonen vår gjelder for venstrevridde på 1800- og tidlig 1900-tall forblir åpent.

\section*{9.}
De to psykologiske tendensene som underligger moderne venstrevridd ideologi kaller vi ''følelser av underlegenhet'' og ''oversosialisering''. Følelser av underlegenhet er karakteristisk for moderne venstrevridd ideologi helhetlig, mens oversosialisering er karakteristisk for bare en viss del av ideologien; men denne delen er svært innflytelsesrik.

\chapter{Følelser av underlegenhet}

\section*{10.}
Med ''følelser av underlegenhet'' tenker vi ikke bare på disse følelsene i en streng forstand, men heller et helt spektrum av relaterte trekk; lav selvtillit, maktesløshet, depressive tendenser, defaitisme, skyldfølelse, selvhat osv. Vi argumenterer for at moderne venstrevridde ofte har slike følelser (muligens mer eller mer undertrykt) og at disse følelsene vil bestemme retningen til moderne venstrevridd ideologi.

\section*{11.}
Når noen tolker nesten alt som blir sagt om vedkommende (eller grupper som vedkommende identifiserer seg med) som støtende, konkluderer vi med at han eller hun føler på underlegenhet eller lav selvtillit. Denne tendensen er utpreget blant minoritetsaktivister, uavhengig av om de tilhører gruppene de forsvarer rettighetene til. De er hypersensitive når det kommer til ordene som brukes for a beskrive minoriteter og alt som blir sagt om disse minoritetene. Ordene ''neger'', ''orientalsk'', ''handikappet'' eller ''høne'' for en afrikaner, asiater, funksjonshemmet person eller kvinne hadde historisk ingen støtende konnotasjon. ''Berte'' eller ''høne'' var bare de feminine versjonene av ''fyr'', ''type'' eller ''gubbe''. De negative konnotasjonene har blitt tillagt disse ordene av aktivistene selv. Noen dyrerettsaktivister har gått så lang som å avvise bruken av ordet ''kjæledyr'' og insisterer på at det skal erstattes med ''dyrefølgesvenn''. Venstrelente antropologer strekker seg langt for å ikke si noe om primitive kulturer som kan tolkes som negativt. De ønsker å erstatte ordet ''primitiv'' med ''analfabet''. De virker nesten paranoide når det kommer til ting som kanskje kan hentyde at en hvilken som helst primitiv kultur er underlegen sammenlignet med vår. (Vi mener ikke å insinuere at primitive kulturer {\em er} underlegne. Vi påpeker bare hypersensitiviteten til venstrelente antropologer.)

\section*{12.}
De som er mest sensitive når det gjelder ''politisk ukorrekt'' ordbruk er ikke den gjennomsnittlige svarte ghettoinnbyggeren, asiatiske innvandreren, kvinnelige overgrepsofferet eller funksjonshemmede, men heller et mindretall av aktivister som i mange tilfeller ikke engang tilhører en ''undertrykt'' gruppe. Disse har en privilegert bakgrunn. Politisk korrekthet står sterkest blant universitetsprofessorer som har sikre jobber med komfortable lønninger, og flertallet av dem er heterofile hvite menn fra middelklasse til øvre middelklasse familier.

\section*{13.}
Mange venstrevridde identifiserer seg sterkt med problemene til grupper som fremstår svake (kvinner), bekjempet (amerikanske indianere), frastøtende (homofile) eller generelt underlegne. Venstrevridde føler selv at disse gruppene er underlegne. De vil aldri innrømme til seg selv at de har disse følelsene, men det er nettopp fordi de ser på disse gruppene som underlegne at de identifiserer seg med problemene deres. (Vi mener ikke å si at kvinner, indianere osv. {\em er} underlegne; vi poengterer bare noe om venstrevridd psykologi.)

\section*{14.}
Feminister er desperate etter å bevise at kvinner er like sterke og kapable som menn. Det er innlysende at de frykter muligheten for at kvinner {\em ikke} er like sterke og kapable som menn.

\section*{15.}
Venstrevridde har en tendens til å hate alt som fremstår som sterkt, godt og suksessfullt. De hater Amerika, vestlig sivilisasjon, hvite menn og rasjonalitet. Grunnene venstrevridde gir for å hate vesten osv., samsvarer tydelig ikke med deres egentlige motiver. De {\em sier} at de hater vesten fordi den er krigersk, imperialistisk, kjønnsdiskriminerende, etnosentrisk osv., men når de samme fenomenene skjer i sosialistiske land eller i primitive kulturer, finner den venstrevridde unnskyldninger for dem. I beste fall vil vedkommende {\em motvillig} si at de eksisterer, mens han i vestens tilfeller vil {\em entusiastisk} påpeke (og ofte overdrive) disse fenomenene. Derfor er det klart at disse feilene ikke er den egentlige grunnen til at den venstrevridde hater Amerika og vesten. Han hater Amerika og vesten på grunn av deres styrke og suksess.

\section*{16.}
Ord som ''selvsikkerhet'', ''selvstendighet'', ''initiativ'', ''foretaksomhet'', ''optimisme'' osv. spiller en liten rolle i liberal og venstrevridd språkbruk. Den venstrevridde er anti-individualistisk og kollektivistisk. Han ønsker at samfunnet skal løse alles problemer, tilfredsstille alles behov og ta vare på alle. Han er ikke typen som har en indre selvsikkerhet når det kommer til hans evne til å løse sine egne problemer eller tilfredsstille sine egne behov. Den venstrevridde misliker idéen om konkurranse fordi han innerst inne føler seg som en taper.

\section*{17.}
Kunstformer som appellerer til moderne venstrelente intellektuelle har en tendens til å fokusere på elendighet, tap og fortvilelse. Ellers har de en ukontrollert tone som fjerner seg fra rasjonell kontroll som om det var ingen håp for å kunne utrette noe ved hjelp av rasjonell kalkulering, og alt som var igjen var å overgi seg til øyeblikkets følelser.

\section*{18.}
Moderne venstrelente filosofer har en tendens til å forkaste fornuft, vitenskap og objektiv virkelighet, og insistere på at alt er kulturelt relativt. Det er sant at man kan stille seriøse spørsmål om grunnlaget til vitenskapelig kunnskap og om hvordan, om i det hele tatt, idéen om objektiv virkelighet kan defineres. Men det er åpenbart at moderne venstrelente filosofer ikke er sindige logikere som systematisk analyserer kunnskapens grunnlag. De er dypt emosjonelt investert i å angripe sannhet og virkelighet. De angriper disse konseptene på grunn av deres egne psykologiske behov. For det første er deres angrep et utløp for fiendtlighet, og i den grad det lykkes, tilfredsstiller det ønsket om makt. Viktigere hater den venstrevridde vitenskap og rasjonalitet fordi de kategoriserer visse synspunkter som sanne (altså suksessrik, overlegen) og andre synspunkter som falske (altså mislykket, underlegen). Den venstrevriddes følelser av underlegenhet går så dypt at han ikke tåler noen form for kategorisering som sier at noe er suksessrikt eller overlegent og andre ting er mislykket eller underlegent. Dette underligger også mange venstrevriddes forkastelse av idéen om psykiske lidelser og nytteverdien til IQ-tester. Venstrevridde har et fiendtligt forhold til genetiske forklaringsmodeller når det kommer til menneskelige evner og adferd, og dette er fordi slike forklaringer har en tendens til å få noen mennesker til å fremstå som sterkere eller svakere enn andre. Venstrevridde foretrekker å gi samfunnet æren eller skylden for et individ sine evner eller mangel på evner. Om en person er ''underlegen'' er det derfor ikke hans feil, men heller samfunnets feil, fordi at han ikke ble oppdratt på en god måte.

\section*{19.}
Den venstrevridde er vanligvis ikke typen som blir påvirket av sine følelser av underlegenhet i den grad at han blir en skrytepave, egoist, mobber, selvhevdende eller et kynisk konkurransemenneske. Denne typen person har ikke helt mistet troen på seg selv. Han har mangler når det kommer til hans følelse av makt og verdighet, men han kan likevel se seg selv som en med kapasitet til å være sterk og hans innsats for å gjøre seg selv sterk forårsaker hans uønskelige oppførsel.\footnote{Vi hevder ikke at alle, eller engang de fleste, mobbere lider av følelser av underlegenhet.} Men den venstrevridde har gått for langt for dette. Hans følelser av underlegenhet sitter så dypt at han ikke kan se på seg selv som individuelt sterk og verdifull. Derfor er den venstrevridde kollektivistisk. Han kan bare føle seg sterk som en del av stor organisasjon eller bevegelse som han identifiserer seg med.

\section*{20.}
Legg merke til de masochistiske tendensene til venstrevridd strategi. Venstrevridde protesterer ved å legge seg ned foran kjøretøy, og de provoserer politiet og rasister slik at de skal misbruke dem osv. Disse strategiene kan ofte være effektive, men mange venstrevridde bruker dem ikke for å nå et bestemt mål. De bruker dem fordi de foretrekker masochistisk strategi. Selvhat er et venstrevridd trekk.

\section*{21.}
Venstrevridde hevder kanskje at deres aktivisme er motivert av medfølelse eller moralske prinsipper, og det er riktig at moralske prinsipper spiller en rolle for den oversosialiserte venstrevridde typen. Men medfølelse og moralske prinsipper kan ikke være hovedmotivene for venstrevridd aktivisme. Fiendtlighet er en for fremtredende del av venstrevridd oppførsel, og det samme er ønsket om makt. Dessuten er mye venstrevridd oppførsel ikke rasjonelt kalkulert til å være til fordel for de menneskene som venstrevridde hevder å hjelpe. Om noen for eksempel mener at kvotering er bra for svarte mennesker, gir det mening å kreve kvotering på en fiendtlig eller dogmatisk måte? Åpenbart ville det vært mer effektivt å gå for en diplomatisk og fleksibel løsning som i det minste ville gitt verbale og symbolske forsikringer til hvite mennesker som mener at kvotering er diskriminerende mot dem. Men venstrevridde aktivister tar ikke en slik strategi i bruk fordi den ikke ville tilfredsstilt deres emosjonelle behov. Å hjelpe svarte mennesker er ikke deres egentlige mål. Raserelaterte problemer fungerer heller som en unnskyldning for at de skal uttrykke sin egen fiendtlighet og behov for makt. Med å gjøre dette skader de faktisk svarte mennesker fordi aktivistenes fiendtlige innstilling overfor den hvite majoriteten har en tendens til å forsterke rasehat.

\section*{22.}
Om samfunnet vårt ikke hadde noen sosiale problemer i det hele tatt, måtte de venstrevridde {\em diktet opp} problemer for å gi seg selv en unnskyldning til å lage bråk.

\section*{23.}
Vi understreker at det foregående hevder ikke å være en nøyaktig beskrivelse av alle som muligens kan defineres som venstrevridd. Det er bare en grov indikasjon på en generell venstrevridd tendens.

\chapter{Oversosialisering}
\section*{24.}
Psykologer bruker begrepet ''sosialisering'' for å beskrive prosessen der barn trenes opp til å tenke og oppføre seg på den måten som samfunnet krever. En person vil sies å være bra sosialisert hvis han tror på og adlyder samfunnets moralske kode og passer bra inn som et velfungerende samfunnsmedlem. Det virker kanskje meningsløst å si at mange venstrevridde er oversosialiserte fordi den venstrevridde blir sett på som en opprører. Uansett kan dette synspunktet forsvares. Mange venstrevridde er ikke like store opprørere som de virker som.

\section*{25.}
Den moralske koden i vårt samfunn er så krevende at ingen kan tenke, føle og handle på en helt moralsk måte. Vi skal for eksempel ikke hate noen, men nesten alle hater noen på et eller annet tidspunkt, uavhengig av om de innrømmer dette overfor seg selv eller ikke. Noen mennesker er så sosialiserte at deres forsøk på å tenke, føle og handle moralsk, pålegger dem en stor byrde. For å unngå skyldfølelse må de hele tiden bedra seg selv når det gjelder sine egne motiver, og de må finne moralske forklaringer for følelser og handlinger som i realiteten ikke har et moralsk opphav. Vi bruker begrepet ''oversosialisert'' for a beskrive slike mennesker. \footnote{Under viktoriatiden led mange mennesker av psykologiske problemer som kom som et resultat av, eller forsøk på undertrykkelse av seksuelle følelser. Freud skal visstnok ha basert sine teorier på denne typen mennesker. I dag har sosialiseringens fokus gått fra sex til aggresjon.}

\section*{26.}
Oversosialisering kan føre til lav selvtillit, en følelse av maktesløshet, defaitisme, skyldfølelse osv. En av de viktigste måtene samfunnet vårt sosialiserer barn er ved å få dem til å føle skam når de handler eller snakker på en måte som går mot samfunnets forventninger. Om dette overdrives, eller om et bestemt barn er spesielt tilbøyelig for slike følelser, ender han opp med å skamme {\em seg}. I tillegg er den oversosialiserte personens tanker og oppførsel mer begrenset av samfunnets forventninger enn en som er lett sosialisert. Flertallet av mennesker er delaktige i betydelige mengder dårlig oppførsel. De lyver, nasker, bryter trafikkregler, unnasluntrer på jobb, hater noen, sier fornærmende ting eller bruker sleipe triks for å komme seg foran andre. Den oversosialiserte personen kan i utgangspunktet ikke gjøre disse tingene, og om han gjør dem, vil han kjenne på følelser av skam og selvhat. Den oversosialiserte personen kan ikke uten skyldfølelse oppleve tanker eller følelser som går mot den alment aksepterte moralen; han kan ikke tenke ''skitne'' tanker. Og sosialisering er ikke bare et spørsmål om moral; vi sosialiseres til å tilpasse oss mange adferdsnormer som ikke alle går under kategorien moral. Ergo blir den oversosialiserte personen holdt med et psykologisk bånd og går gjennom livet gående på veier som samfunnet har laget for ham. For mange oversosialiserte personer forårsaker dette en følelse av begrensning og maktesløshet som kan være svært vanskelig. Vi foreslår at oversosialisering er blant de mer seriøse grusomhetene mennesker kan gjøre mot hverandre.

\section*{27.}
Vi argumenterer for at en veldig viktig og innflytelsesrik del av den moderne venstresiden er oversosialisert og at deres oversosialisering er av stor viktighet når det kommer til å bestemme retningen til moderne venstrevridd ideologi. Venstrevridde av den oversosialiserte typen har en tendens til å være intellektuelle eller medlemmer av øvre middelklasse. Legg merke til at intellektuelle ved universiteter\footnote{Spesialister innenfor ingeniørfag eller ''seriøs'' vitenskap er ikke nødvendigvis inkludert her.} er den mest sosialiserte, og også den mest venstrevridde delen av samfunnet.

\section*{28.}
Den venstrevridde av den oversosialiserte typen prøver å fjerne seg fra det psykologiske båndet og hevde sin automoni ved å gjøre opprør, men vanligvis er han ikke sterk not til å gjøre opprør mot samfunnets mest grunnleggende verdier. Generelt sett er målene til dagens venstrevridde {\em ikke} i konflikt med den alment aksepterte moralen. Tvert i mot vil venstresiden ta et akseptert moralsk prinsipp, internalisere det, og så kritisere storsamfunnet for å bryte det prinsippet. Eksempler er raselikhet, likestilling mellom kjønnene, å hjelpe fattige, fred i stedet for krig, ikkevold generelt, frihet for å uttrykke seg slik man vil og vennlighet mot dyr. Mer grunnleggende individets plikt til å tjene samfunnet, og samsfunnets plikt til å ta seg av individet. Alle disse verdiene har vært dypt rotfestet i vårt samfunn (eller i det minste i middel og øvre klasser\footnote{Det er mange individer i middel og øvre klasser som motsetter seg noen av disse verdiene, men vanligvis er deres motstand mer eller mindre skjult. Slik motstand kommer frem i massemediene i en veldig begrenset grad. Hoveddelen av propaganda i vårt samfunn er til fordel for fastsatte normer. Hovedgrunnen til at disse verdiene har blitt så å si de offisielle verdiene i vårt samfunn, er fordi de er nyttige for det industrielle samfunn. Vold er frarådet fordi det forstyrrer systemets funksjon. Rasisme er frarådet fordi etniske konflikter også forstyrrer systemet, og diskriminering kaster bort talentene til minoriteter som kunne vært nyttig for systemet. Fattigdom må ''kureres'' fordi underklassen forårsaker problemer for systemet og kontakt med underklassen senker moralen til de andre klassene. Kvinner blir oppfordret til å ha karrierer fordi deres talenter er nyttige for systemet, og viktigere fordi ved å ha vanlige jobber blir de bedre integrert inn i systemet og knyttes direkte til systemet fremfor familiene deres. Dette hjelper med å svekke samholdet i familien. (Systemets ledere sier at de ønsker å styrke familien, men det de egentlig mener er at de ønsker at famlien skal fungere som et effektivt verktøy for å sosialisere barn etter systemets behov. Vi argumenterer i avsnittene 51 og 52 for at systemet ikke har råd til å la familien eller andre små sosiale grupper bli sterke eller selvstendige.)}) lenge. Disse verdiene er eksplisitt eller implisitt uttrykt eller forutsett i mesteparten av stoffet som presenteres til oss av mediene og utdanningsinstitusjoner. Venstrevridde, spesielt de av den oversosialiserte typen, vil vanligvis ikke gjøre opprør mot disse prinsippene, men heller rettferdiggjøre deres fiendtlighet overfor samfunnet ved å hevde (med en viss grad av sannhet) at samfunnet ikke lever opp til disse prinsippene.

\section*{29.}
Her er en illustrasjon på hvordan den venstrevridde viser hvor tett han er knyttet til samfunnets konvensjonelle normer imens han later som å gjøre opprør mot dem. Mange venstrevridde kjemper for kvotering, å flytte svarte mennesker inn i presisjefylte jobber, forbedret utdanning i svarte skoler og mer penger for slike skoler; den svarte ''underklassens'' levemåte ser de på som en skam. De ønsker å integrere den svarte mannen inn i systemet ved å gjøre han til en forretningsmann, advokat eller en vitenskapsmann akkurat som hvite mennesker i den øvre middelklasse. De venstrevridde vil svare med å si at det siste de ønsker å gjøre den svarte mannen til en kopi av den hvite mannen, og at de ønsker å bevare afroamerikansk kultur. Men hva består denne bevaringen av? Den kan knapt bestå av mer enn å spise afroamerikansk mat, høre på afroamerikansk musikk, gå med afroamerikanske klær og å gå til en afroamerikansk kirke eller moské. Med andre ord kan det bare uttrykkes i overfladiske tilfeller. På alle {\em vesentlige} måter vil venstrevridde av den oversosialiserte typen gjøre slik at den svarte mannen må tilpasse seg idealene til den hvite middelklassen. De ønsker at han skal studere tekniske fag, bli en leder eller vitenskapsmann og å bruke livet sitt på å klatre karrierestigen for å bevise at svarte mennesker er like gode som hvite. De ønsker å gjøre svarte fedre ''ansvarlige'', svarte gjenger ikkevoldelige osv. Men dette er det industrielt-teknologiske systemets nøyaktive verdier. Systemet kunne ikke brydd seg mindre om hva slags musikk man lytter til, hva slags klær man går med eller hvilken religion man tror på så lenge man går på skolen, har en respektabel jobb, klatrer karrierestigen, er en ''ansvarlig'' forelder, er ikkevoldelig osv. Effektivt sett vil den venstrevridde integrere den svarte mannen inn i systemet og få han til å internalisere dets verdier, uavhengig av hvor mye han måtte fornekte dette.

\section*{30.}
Vi hevder absolutt ikke at venstrevridde, selv av den oversosialiserte typen, {\em aldri} gjør opprør mot vårt samfunns grunnleggende verdier. Åpenbart gjør de dette noen ganger. Noen oversosialiserte venstrevridde har gått så langt som å gjøre opprør mot et av samfunnets viktigste prinsipper ved bruke vold. Ifølge deres egne ord, er vold for dem en form for ''frigjøring''. Med andre ord, ved å bruke vold bryter de gjennom de psykologiske begrensningene som er programmert inn i dem. Fordi de er oversosialisert har disse begrensningene vært mer omfattende for dem enn for andre; ergo deres ønske om å bryte ut av dem. Men de vil vanligvis rettferdiggjøre sitt opprør med verdier som er vanlige. Om de bruker vold vil de hevde at de kjemper mot rasisme og lignende.

\section*{31.}
Vi innser at mange innvendinger kan komme mot den foregående beskrivelsen av venstrevridd psykologi. Den virkelige situasjonen er komplisert, og noe som kan nærme seg en fullstendig forklaring ville brukt flere bøker selv om den nødvendige dataen var tilgjengelig. Vi hevder bare å ha indikert på en veldig grov måte de to viktigste tendensene i psykologien til moderne venstrevridd ideologi.

\section*{32.}
Den venstrevriddes problemer er også en indikasjon på samfunnets problemer helhetlig. Lav selvtillit, depressive tendenser, og defaitisme er ikke begrenset til venstresiden. Selv om de er spesielt synlige på venstresiden, er de utbredt i vårt samfunn. Og dagens samfunn prøver å sosialisere i en større grad enn noe tidligere samfunn. Vi blir til og med fortalt av eksperter hvordan vi skal spise, trene, ha sex, oppdra våre barn osv.

\chapter{Maktprosessen}
\section*{33.}
Mennesker har et behov (mest sannsynlig biologisk) for noe vi kaller maktprosessen. Dette er nærmt beslektet ønsket om makt (som er bredt anerkjent), men ikke helt det samme. Maktprosessen har fire elementer. De tre tydeligste av disse kaller vi mål, innsats og måloppnåelse. (Alle trenger å ha mål der oppnåelsen krever innsats, og trengs for å oppnå minst noen av hans mål.) Det fjerde elementet er vanskeligere å definere og er kanskje ikke nødvendig for alle. Vi kaller det autonomi og vi vil diskutere det senere (avsnitt 42-44).

\section*{34.}
Tenk på en hypotetisk mann som kan få hva han vil bare ved å ønske det. En slik mann har makt, men han vil utvikle seriøse mentale problemer. I begynnelsen vil han ha mye moro, men snart vil han bli høyst lei og demoralisert. Etterhvert kan han bli klinisk deprimert. Historien viser oss at aristokratier med mye fritid har en tendens til å forfalle. Dette er ikke tilfelle i kjempende aristokratier som må streve for å beholde sin makt. Men artistokratier som er sikre og har mye fritid, og som ikke trenger å anstrenge seg blir vanligvis kjedsom, hedonistisk og demoralisert, selv om de har makt. Dette viser at makt ikke er nok. Man må ha mål som man kan bruke sin makt til å nå.

\section*{35.}
Alle har mål; om ingenting annet, å få tak i det nødvendige man trenger for å leve: mat, vann, klær og husly (de to siste vil variere etter klima). Men aristokraten med mye fritid får tak i disse tingene uten innsats. Ergo hans kjedsomhet og demoralisering.

\section*{36.}
Å ikke nå viktige mål fører til død om målene er fysiske nødvendigheter,

\chapter{Erstatningsaktiviteter}

\chapter{Autonomi}

\chapter{Årsaker til samfunnsmessige problemer}

\chapter{Forstyrrelse av maktprosessen i det moderne samfunn}

\chapter{Hvordan noen mennesker tilpasser seg}

\chapter{Forskeres motiver}

\chapter{Frihetens natur}

\chapter{Noen historiske prinsipper}

\chapter{Industrielt-teknologisk samfunn kan ikke reformeres}

\chapter{Begrensning av frihet er uunngåelig i det industrielle samfunn}

\chapter{De 'negative' aspektene av teknologi kan ikke adskilles fra de 'positive'}

\chapter{Teknologi er en mektigere samfunnsmessig kraft enn ønsket om frihet}

\chapter{Enklere samfunnsmessige problemer har vist seg å være uoverkommelige}

\chapter{Revolusjon er enklere enn reform}

\chapter{Kontroll av menneskelig adferd}

\chapter{Menneskeheten ved et veiskille}

\chapter{Menneskelig lidelse}

\chapter{Fremtiden}

\chapter{Strategi}

\chapter{To former for teknologi}

\chapter{Faren for venstrevridd ideologi}

\chapter{Sluttord}

\end{document}
